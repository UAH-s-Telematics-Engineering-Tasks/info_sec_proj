Los troyanos son el primer paso de muchos ataques informáticos que se llevan a cabo. Su nombre se basa en la famosa leyenda de la mitología griega del caballo de Troya, una enorme estructura de madera que tenía el aspecto del propio animal y que, a primera vista, parecía que no entrañaba ningún peligro, pero en su interior albergaba a varios soldados que empleaban este método para esconderse y no ser descubiertos. Basándose en esta idea, este tipo de malware consiste en un programa que en primera instancia parece inofensivo, hasta el punto de poder resultar atractivo, por lo que las víctimas que lo padecen no son capaces de percibir su peligrosidad debido a su apariencia amigable. Sin embargo, una vez que se descarga y llega a ejecutarse, se pone en marcha su cometido real que no es otro que el de infectar la máquina que lo aloja para posibilitar que el atacante que se encuentra detrás de él se haga con el control de la misma, todo ello de manera transparente y prácticamente imperceptible para el cliente que es dañado. Tras adueñarse del equipo y teniendolos como punto de partida, posteriormente se puede ser capaz de obtener multitud información de diverso tipo del mismo además de llevar a cabo acciones más perjudiciales. De este modo, se puede llegar a lograr manejar archivos que se albergan en la máquina, copiándolos, eliminándolos o enviándolos, robar datos privados, controlar los procesos que están teniendo lugar, obtener un registro de las pulsaciones que se producen así como de lo que se está visualizando en la pantalla e incluso poder instalar nuevos softwares para que sean estos los que realicen actuaciones de mayor relevancia. De forma similar, pueden también implementar diversos mecanismos que faciltan la conexión al ordenador desde otros lugares como, por ejemplo, las denominadas puertas traseras o backdoors. Por tanto, el cometido principal de los troyanos es ocultarse a la víctima y esconder su funcionalidad maliciosa para, a continuación, después de lograr ponerse en funcionamiento al ejecutarse el programa que alberga el mismo, posibilitar acciones maliciosas llevadas a cabo por él mismo o por otros programas o individuos a los que facilita el acceso a la máquina infectada.

Una de las funcionalidades de las que hacen uso estos troyanos son las backdoors o puertas traseras. Estos recursos, tal y como su nombre bien indica, permiten evadir los rigurosos controles de seguridad existentes para poder acceder a un ordenador y consienten, por medio de estos pasadizos en los que no es necesario llevar a cabo verificación alguna, la entrada a la máquina de forma muy sencilla y asequible, lo cual simplifica que el atacante externo de un equipo alcance el mismo sin ninguna dificultad para obtener el control del mismo y poner en práctica actos malintencionados y nocivos para el dueño del dispositivo. A través de estos abertura secreta se consigue poner en práctica todas las actuaciones maléficas comentadas con anterioridad.

Los rootkits son un tipo de malware que afecta al sistema operativo que se aloja en el ordenador. De este modo, logran modificar la configuración del mismo para posibilitar la presencia de mecanismos maliciosos sin que el usuario de la máquina logre percatarse de ello. Así, consigue ocultar los mismos para que se ejecuten y lleven a cabo su cometido a escondidas del dueño del equipo, de la lista de procesos en funcionamiento e incluso de los propios antivirus. Esto procedimiento camufla también los archivos que sustentan los mencionados programas perjudiciales. Es posible, tambien, elaborar accesos imperceptibles del tipo backdoor para entrar de forma muy rápida y fácil al dispositivo desde el exterior y poder controlarlo para obtener información comprometida o poner en marcha acciones perversas tal y como sucede con los troyanos también. Por consiguiente, esta labor de enmascaramiento que llevan consigo dificultan en gran medida su detección. Este ataque puede dirigirse incluso al núcleo del sistema operativo o kernel lo que conlleva unos riesgos aún mayores debido a la importancia de esta parte para el correcto funcionamiento de todo el conjunto del ordenador.