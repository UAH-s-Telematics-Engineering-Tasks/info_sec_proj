\section{INTRODUCCIÓN}
	Con el aumento del uso de las nuevas tecnologías y de Internet, se han incrementado los ataques e intrusiones a equipos e información poniendo en compromiso la integridad, confidencialidad o disponibilidad de los recursos del sistema. Esto es debido a la creciente exposición ante ataques por el entorno en el que son utilizados y el mayor conocimiento sobre el funcionamiento de los sistemas que hacen que los intrusos estén mejor preparados para explotar vulnerabilidades. Gracias al fácil acceso y transmisión de información que tenemos hoy en día, es posible incluso explotarlas desde el mismo momento en el que se da a conocer un software (conocidas como vulnerabilidades zero-day), aunque no se haya tenido acceso directo a la información del desarrollador.

	A este tipo de software cuyo objetivo es infiltrarse o dañar una computadora o un sistema de información sin el consentimiento del propietario se le conoce como malware. Dependiendo de los efectos deseados de ese software, el malware se puede clasificar en varios tipos. En el caso de nuestro grupo, nos centraremos en el malware oculto.

	Esta ocultación de software malicioso se consigue normalmente empleando un método de ofuscación del código del software de forma que oculta su flujo de control y su estructura enmascarando así su verdadera función de manera que la víctima no es consciente.
	
	Existen sistemas de detección de intrusiones (IDS) que consisten en hardware o software especializado en la automatización de la monitorización de redes y sistemas en busca de evidencias de violación de una determinada política de seguridad. Una adecuada elección y configuración del IDS puede incrementar en gran medida la seguridad de una red o sistema.

	A continuación desarrollaremos de forma independiente cada uno de los casos más globales de malware oculto.