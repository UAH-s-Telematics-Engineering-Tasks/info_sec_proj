\section{Tipos de malware}
	Como hemos comentado, el mundo de los malwares es muy amplio y diverso y nos encontramos muchos tipos diferentes de ellos. En nuestro caso nos centraremos en el malware oculto, el cual también abarca muchas variantes y posibilidades y que acotaremos centrándonos en los troyanos, las puertas traseras o \textit{backdoors}, los rootkits y los Drive-by downloads, cuyas cualidades más reseñables pasaremos a exponer de inmediato.

	\subsection{Troyanos}
		Los troyanos son el primer paso de muchos ataques informáticos que se llevan a cabo. Su nombre se basa en la famosa leyenda de la mitología griega del caballo de Troya, una enorme estructura de madera que tenía el aspecto del propio animal y que, a primera vista, parecía que no entrañaba ningún peligro, pero que en su interior albergaba a varios soldados que empleaban este procedimiento para esconderse y no ser descubiertos.

		Basándose en esta idea, este tipo de malware consiste en un programa que en primera instancia parece inofensivo, hasta el punto de poder resultar atractivo, por lo que las víctimas que lo padecen no son capaces de percibir su peligrosidad debido a su apariencia amigable. Sin embargo, una vez que se descarga y llega a ejecutarse, se pone en marcha su cometido real que no es otro que el de infectar la máquina que lo aloja para posibilitar que el atacante que se encuentra detrás de él se haga con el control de la misma, todo ello de manera transparente y prácticamente imperceptible para el cliente que es dañado.

		Tras adueñarse del equipo y teniendo a estos programas maléficos como punto de partida, posteriormente se puede ser capaz de obtener multitud información de diverso tipo del mismo además de llevar a cabo acciones más perjudiciales.

		De este modo, en un primer momento, los troyanos se empleaban para realizar el mayor daño posible y sin ningún miramiento, tratando de formatear el ordenador envenenado o eliminar archivos del sistema. Sin embargo, no consiguieron la trascendencia y notoriedad que buscaban, limitados por la imposibilidad de propagarse y multiplicarse por si mismos, algo que les diferencia de los virus y los gusanos informáticos y que les frenaba en gran medida, por lo que pasaron a buscar nuevos objetivos en los que enfocarse.

		Así, en la actualidad se puede llegar a lograr manejar archivos que se albergan en la máquina, copiándolos, eliminándolos o enviándolos, robar datos privados tales como números de tarjetas bancarias, controlar los procesos que están teniendo lugar, denegar el servicio, obtener un registro de las pulsaciones que se producen así como de lo que se está visualizando en la pantalla e incluso poder instalar nuevos softwares para que sean estos los que realicen actuaciones de mayor relevancia.

		De forma similar, pueden también implementar diversos mecanismos que faciltan la conexión al ordenador desde otros lugares como, por ejemplo, las denominadas puertas traseras o backdoors. Para realizar estas técnicas, los troyanos son capaces de llevar a cabo la apartura de puertos de comunicaciones de tal forma que se consigue materializar el objetivo de abrir un punto a través del cual sea posible alcanzar el equipo envenenado.

		Por tanto, el cometido principal de los troyanos es ocultarse a la víctima y esconder su funcionalidad maliciosa para, a continuación, después de lograr ponerse en funcionamiento al ejecutarse el programa que lo alberga, posibilitar acciones malvadas llevadas a cabo por él mismo o por otros sistemas o individuos a los que facilita el acceso a la máquina infectada.

		Este tipo de malware puede situarse en páginas que no son muy frecuentadas y que, además, resultan hasta cierto punto desconocidas, en las cuales incluso pueden encontrarse aplicaciones ejecutables que, a primera vista, no parecen entrañar ningún peligro y pueden ser, del mismo modo, de confianza y legítimas. También pueden provenir de descargas realizadas en redes P2P (peer to peer), en las cuales, en ocasiones, la seguridad no resulta uno de sus puntos fuertes por cuanto son accesibles por todo el mundo y es posible alojar archivos en ellas sin ejercer un control exhaustivo sobre el contenido de los mismos.

		Los troyanos se componen de dos partes de vital importania para su correcto funcionamiento e implementación. Así, constan de un cliente y un servidor, cada uno de los cuales se encuentra alojado en la máquina atacada y atacante, respectivamente, o viceversa en función del procedimiento seguido para el ataque, el cual detallaremos con posterioridad junto con las características de los dos extremos de la interacción para concretar la intrusión.

	\subsection{Backdoors}
		Una de las funcionalidades de las que hacen uso estos troyanos son las puertas traseras o \textit{backdoors}, por lo tanto, no constituyen un mecanismo independiente de ataque informático sino que más bien son un complemento a estos últimos.

		Los fines de las backdoors pueden resultar muy diversos y variados y no todos tienen por qué ser dañinos y perjudiciales. Pueden derivar de múltiples orígenes, no únicamente relacionadas a los troyanos, pues puede darse la situación de que los desarrolladores de sistemas y aplicaciones se olvidaran de eliminarlas o bloquearlas o, simplemente, continúen existiendo para efectuar determinadas tareas legales de mantenimiento o actualización de forma transparente al usuario, sin que se requiera de su participación y facilitando su experiencia y uso, como ocurre, por ejemplo, en equipos que reciben actualizaciones sin que su dueño sea consciente de ello u otros con determinados problemas técnicos los cuales son subsanados mediante el acceso remoto de un especialista en su reparación sin que propietario deba hacer nada.

		En cambio, en nuestro estudio nos centraremos en aquellas que están más enfocadas a realizar acciones negativas para la víctima y que atenten contra su privacidad e intimidad.

		Estos recursos, tal y como su nombre bien indica, favorecen esquivar los rigurosos controles de seguridad existentes para poder acceder a un ordenador y consienten, por medio de estos pasadizos en los que no es necesario llevar a cabo verificación alguna, la entrada a la máquina de forma muy sencilla y asequible, lo cual simplifica que el atacante externo de un equipo alcance el mismo sin ninguna dificultad para obtener el mando del mismo y poner en práctica actos malintencionados y nocivos para el dueño del dispositivo. A través de esta abertura secreta se consiguen ejecutar todas las actuaciones maléficas comentadas con anterioridad.

		Las backdoors se originan como consecuencia de la infeccción de una máquina por parte de un troyano, pero, tambien, pueden producirse a través de diversos fallos de seguridad que originen ciertos resquicios, los cuales son aprovechados por los atacantes para desarrollar su objetivo ilícito.

		Las puertas traseras, tal y como hemos mencionado previamente, se implementan abriendo puertos de comunicaciones que son una implementación lógica a la cual se dirige determinada información procedente de un origen concreto y que, por consiguiente, permiten establecer conexiones con otras máquinas existentes en la inmensa red de Internet.

		En algunas situaciones el mecanismo de las backdoors o puertas traseras se empleada para crear botnets, cuyo término proviene de la convinación de las palabras inglesas \textit{robot} y \textit{network} (robot y red en español, respectivamente), y que consisten en redes constituídas por equipos que ya han sido infectados y son controlados todos ellos en su conjunto por un atacante.

		Los fines de este entramado se centran principalmente en generar ataques de denegación de servicio que impidan el acceso a determinadas páginas web por parte de otras personas. Para ello, hacen uso de la embergadura y capacidad del conjunto para colapsar los servidores de estos sitios efectuando un gran número de solicitudes, desencadenado el bloqueo y la caída de los mismos.

		También se utilizan estas redes de robots para enviar masivamente los popularmente conocidos correos electrónicos basura o \textit{spam}, que no son más que emails con información poco relevante y que, además, pueden suponer un peligro para la seguridad de nuestros aparatos.

	\subsection{Rootkits}
		Los rootkits son un tipo de malware que afecta al sistema operativo que se aloja en el ordenador. De este modo, logran modificar la configuración del mismo para posibilitar la presencia de mecanismos maliciosos sin que el usuario de la máquina logre percatarse de ello. Al igual que ocurre con los troyanos, no son capaces de propagarse automáticamente.

		Así, consigue ocultar los mismos para que se ejecuten y lleven a cabo su cometido a escondidas del dueño del equipo, de la lista de procesos en funcionamiento e incluso de los propios antivirus. Esto procedimiento camufla también los archivos que sustentan los mencionados programas perjudiciales además de las conexiones a través de la red que estos desarrollan y ciertas configuraciones establecidas.

		Es posible, tambien, elaborar accesos imperceptibles del tipo backdoor para entrar de forma muy rápida y fácil al dispositivo desde el exterior y poder controlarlo para obtener información comprometida o poner en marcha acciones perversas tal y como sucede con los troyanos también.

		Por consiguiente, esta labor de enmascaramiento que llevan consigo dificultan en gran medida su detección.

		Para su instauración en un equipo se requieren permisos de escritura en él, los cuales pueden ser adquiridos por el atacante recurriendo a alguna vulnerabilidad presente en el aparato o mediante el conocimiento de la clave de desbloqueo valiendose de los múltiples disponibles para tal fin.

		Dependiendo de las características que ostenten así como su estilo de proceder se diferencian, según su persistencia, por un lado, los rootkit persistentes que se activan cada vez que se inicia todo el sistema y, por otra parte, los no persistentes, que no operan tras llevarse a cabo un reinicio del dispositivo.

		Considerando el modelo de ejecución que implementan encontramos aquellos que se limitan al ambiente de usuario pues se centran en los programas existentes mientras que otros se dirigir, incluso, al núcleo del sistema operativo o \textit{kernel} lo que conlleva unos riesgos aún mayores debido a la importancia y trascendencia de esta parte para el correcto funcionamiento de todo el conjunto del ordenador.

		En origen, esta actuación estaba destinada a entornos de tipo UNIX, de ahí su nombre pues en dicho ámbito al administrador del sistema se le denomina también como \textit{robor}. Sin embargo, con el paso del tiempo buscaron un nuevo enfoque y se interesaron por máquinas equipadas con Windows en las que su comportamiento era exactamente el mismo.

		En cambio, no todo son conductas ilegales ya que se hallan contextos para los que sí son legítimos como, por ejemplo, la supervisión de los empleados de una empresa o corporación, la protección de datos intelectuales o frente a errores de una persona como pueden ser borrados accidentales.

	\subsection{Drive-by Download}
		El malware oculto de tipo Drive-by download es uno de los métodos de infeccción más simples que puede existir. Su simplicidad radica en el hecho de que el usuario que va a ser víctica de él no debe realizar ninguna acción excepcional, pulsar en sitio alguno o aceptar ninguna descarga.

		El ataque se materializa cuando el individuo se adentra en una página web y, de manera inmediata y sin que muestre evidencia de ello, se inica la bajada del programa pernicioso.

		Para ello, en esta técnica se esconde el software nocivo de diversas formas. Por una parte, este se incluye entre el propio código HTML de la página web. De forma similar, puede ocurrir que se resguarde detrás de los anuncios publicitarios que aparecen en ciertos de estos sitios visitados, aprovechando determinadas carencias que puedan presentar complementos como Java, Flash u otros de idénticas características.

		En estos casos, suele suceder que el código camuflado puede constituir tan solo una pequeña parte de todo el malware en conjunto, por lo que, una vez efectuado con éxito el ataque, se solicita de forma automática el contenido restante del mismo al servidor externo que lo alberga.

		Las páginas web que contienen estos programas pueden ser accedidas explícitamente por el usuario al ser conocedor de ellas y realizar su busqueda en un navegador web, tal y como puede suceder con entornos de escasa seguridad relacionados con recursos un tanto peculiares como, por ejemplo, pornográfico, o bien, ser conducido a ellas a través de la recepción de correos electrónicos que le invitan a ello y ventanas emergentes. En estos últimos mecanismos, el aspecto familiar y aparentemente inofensivo favorece que la víctima en ningún momento llegue a sospechar de ellas.

Bibliografía
http://virustroyanosinformatica.blogspot.com/
https://es.wikipedia.org/wiki/Troyano_(inform%C3%A1tica)
https://es.wikipedia.org/wiki/Rootkit
https://www.welivesecurity.com/la-es/2015/04/17/que-es-un-backdoor/
https://www.pandasecurity.com/es/security-info/rootkit/
https://www.pandasecurity.com/es/security-info/back-door/
https://www.kaspersky.es/resource-center/threats/botnet-attacks
https://www.redeszone.net/2018/11/04/drive-by-tipos-funciona-protegernos/